\documentclass{article}
\setlength{\parskip}{2.0mm}

\usepackage{listings}
\lstset{
breaklines=true,
basicstyle=\footnotesize,
tabsize=2
}

\usepackage{a4wide}
\usepackage{listings}
\usepackage{color}
\usepackage{graphicx}
\usepackage{float}
\usepackage{amsmath}
\usepackage{subfig}
\usepackage{cite}
\usepackage{url}
\usepackage{amsmath}

\begin{document}

\title{Robotics - Kalman filter}

\author{Oleg Iegorov, 
\and Jander Nascimento}

\maketitle

\section{Wavefront}


\section{Wavefront Algorithm}

The algortihm we are using to find the shortest path from the start
position to the goal position is based on the classical Dijkstra's
algorithm. The cells of the workspace grid are the vertices of the
corresponding graph and each cell has maximum 4 neighbors.

Given the start position we compute the path length to all other cells
of the workspace. Then, the goal position is taken and if the path from
start to goal exists, the shortest one is returned.

To calculate the path length from the given cell to all others we have
to maintain an additional array \emph{visited[][]}. It keeps track of
those cells for which the shortest path has been already computed.  A
\emph{navigation[][]} array stores the distance from the start position for
each cell.

The main algorithm is thus the following:

\begin{lstlisting}
visited[start_x][start_y] = 1;
navigation[start_x][start_y] = 0;
for each neighbor of the start position
  put navigation[][] value to 0 + 1;
for each cell in the grid
  current = unvisited cell with the smallest navigation value;
  nav_value = navigation[current_x][current_y];
  mark it as visited;
  for each neigbor of this cell
    if (navigation > nav_value)
      navigation = nav_value + 1;
\end{lstlisting}


\dots\dots path finding\dots\dots

\section{Specification}

The application is coded with a 5x6 matrix. The workspace is mapped inside the code, so is not possible to change the workspace without re-compile the code. Although is easy to compile the application.

\section{How to Compile+Run}

To compile the application simply type {\it make} and then simply type the name of the application, in our case is \emph{wave}.

\end{document}


