\documentclass{article}
\setlength{\parskip}{2.0mm}

\usepackage{listings}
\lstset{
breaklines=true,
basicstyle=\footnotesize,
tabsize=2
}

\usepackage{a4wide}
\usepackage{listings}
\usepackage{color}
\usepackage{graphicx}
\usepackage{float}
\usepackage{amsmath}
\usepackage{subfig}
\usepackage{cite}
\usepackage{url}
\usepackage{amsmath}

\begin{document}

\title{Robotics - Kalman filter}

\author{Oleg Iegorov, 
\and Jander Nascimento}

\maketitle

\section{Wavefront Algorithm}

The algorithm we are using to find the shortest path from the start
position to the goal position is based on the classical Dijkstra's
algorithm. The cells of the workspace grid are the vertices of the
corresponding graph and each cell has maximum 4 neighbors.

Given the start position we compute the path length to all other cells
of the workspace. Then, the goal position is taken and if the path from
start to goal exists, the shortest one is returned.

To calculate the path length from the given cell to all others we have
to maintain an additional array \emph{visited[][]}. It keeps track of
those cells for which the shortest path has been already computed.  A
\emph{navigation[][]} array stores the distance from the start position for
each cell.

The main algorithm is thus the following:

\begin{lstlisting}[language=C]
...
visited[start_x][start_y] = 1;
navigation[start_x][start_y] = 0;
for each neighbor of the start position
  put navigation[][] value to 0 + 1;
for each cell in the grid
  current = unvisited cell with the smallest navigation value;
  nav_value = navigation[current_x][current_y];
  mark it as visited;
  for each neighbor of this cell
    if (navigation > nav_value)
      navigation = nav_value + 1;
...
\end{lstlisting}

\section{Shortest path selection}

After apply the navigation function, we have a connected graph in which all the nodes receives the value that represents the distance between itself and the goal. Thus, as long as the goal do not change, the navigation function result stays the same.
 
The idea behind the the shortest path selection can be represented in the Equations \ref{eq:neighbor} and \ref{eq:path}. Where given a $x$ and $y$ (which we call the start position) we navigate through the graph always selecting in each node the shortest values among the neighbors (Equation \ref{eq:path}).

Then we perform the same operation in this selected node until the value reached is 0(which represents the cost of our goal).
 
\begin{equation}
 neighbor(x,y)=\{n(x,y+1),n(x,y-1),n(x+1,y),n(x-1,y)\}
\label{eq:neighbor}
\end{equation}

\begin{equation}
 \operatorname{arg\,min}cost(neighbors(x,y)) \forall x,y \in N
\label{eq:path}
\end{equation}


\section{Specification}

The workspace dimension is coded with a 5x6 matrix, which represents the workspace where the navigation function is going to be applied. 

As the workspace is statically written inside the code, it is not possible to change the workspace without re-compile the code. Although is easy to compile the application.

\section{How to Compile \& Run}

To compile the application simply type {\it make} and then run the application.

The code is also available at the address \url{https://code.google.com/p/jfrobotics/source/browse/trunk/wavefront/}.

\end{document}


