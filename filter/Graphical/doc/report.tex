\documentclass{article}

\usepackage{listings}
\usepackage{color}
\usepackage{graphicx}
\usepackage{float}
\usepackage{amsmath}
\usepackage{subfig}
\usepackage{cite}
\usepackage{url}
\usepackage{amsmath}

\begin{document}

\title{Robotics - Localization and Prediction}

\author{Asan Agibetov, 
\and Jander Nascimento}

\maketitle

\section{Localization of a mobile robot}

\subsection{Estimating with Kalman filter}
%Explain how kalman works and why it can be applied in the case of the robot

\subsection{Testing data sets}
%run in couple data sets and explain what is going on
\emph{Q} and \emph{R} are respectively related with the confidence in the action and in the observation. This means that the values will converge to real values faster.

\section{Detection and Tracking of a moving person}

\subsection{Explain Kalman for DATMO}
%explain how the kalman is used for tracking

\subsection{Testin data sets}
%run in couple data sets and explain what is going on

\end{document}


