\documentclass{article}
\setlength{\parskip}{2.0mm}

%\usepackage{a4wide}
\usepackage{listings}
\usepackage{color}
\usepackage{graphicx}
\usepackage{float}
\usepackage{amsmath}
\usepackage{subfig}
\usepackage{cite}
\usepackage{url}
\usepackage{amsmath}

\begin{document}

\title{Robotics - Kalman Filter}

\author{Oleg Iegorov}

\maketitle

\section{Localization of a mobile robot}

The Kalman filter is used to predict the state of the robot at a particular
moment of time \emph{t} knowing an estimated state at time \emph{t-1}
and some additional information about the system. It is a recursive
algorithm that uses prediction and estimation/correction routings as its core
elements.

\begin{figure}[H] \centering
  \includegraphics[scale=0.3]{routings.png}
\end{figure}

The formulas that describe Kalman Filter are the following:

\begin{equation} \label{eq:prediction}
  Prediction\;:\; 
  \left\{ 
    \begin{array}{ c }
      \mu^{-}_{t+1} = \mu_t + A_t \\
      \Sigma^{-}_{t+1} = \Sigma_t + Q_t
    \end{array}
  \right.
\end{equation}

\begin{equation} \label{eq:kalman_gain}
  Kalman\;gain\;:\; K_t = \frac{\Sigma^{-}_{t+1}}{\Sigma^{-}_{t+1} + R_t}
\end{equation}

\begin{equation} \label{eq:estimation}
  Estimation\;:\; 
  \left\{ 
    \begin{array}{ c }
      \mu_{t+1} = \mu^{-}_{t+1} + K_t(O_t - \mu^{-}_{t+1}) \\
      \Sigma_{t+1} = (1 - K_t)\Sigma^{-}_{t+1}
    \end{array}
  \right.
\end{equation}
\\
\begin{tabbing}
\hspace{10mm}where \hspace{5mm} \= $\mu^{-}_{t+1}$ is an \emph{a priori} state estimation \\
\> $\mu_{t+1}$ is an \emph{a posteriori} state estimation\\
\> $A_t$ --- action at time t\\
\> $Q_t$ --- noise associated with an action estimation\\
\> $R_t$ --- noise associated with an observation estimation\\
\> $O_t$ --- observation at time t
\end{tabbing}

These formulas were coded in the file Kalman.cpp

The value for action $A_t$ from equation \ref{eq:prediction} was statically
specified in \emph{Localization.cpp} to be equal 16 in all the
experiments. The values for $Q_t$, $R_t$ and initial position and
standard deviation were also specified in this file:

\begin{lstlisting}
float a = 16;
// Initialization of Kalman Filter
float init_q = 4.0;
float init_r = 4.0;
float init_x = 16;
float init_std = 3;
\end{lstlisting}

As an observation $O_t$ at each timestep we take the value of a beam
which is perpendicular to OY axis. As the total number of beams is 768
the beam of interest is $\frac{768}{2}=384$th

The tests were done on the data sets from \emph{DataSets/perfectcase/}
and\\
\emph{Datasets/changemotion/} directories.

\begin{enumerate}
  \item Perfect case\\
    In the first data set the robot moves in one
    direction with constant action 16. As we have statically
    specified the same value of action, the predicted position quickly
    converges to the observed one over time. After only 4 timesteps the
    predicted value is almost equal to the observed value. The change of
    any parameters of Kalman Filter does not change anything. This
    happens because after several timesteps the predicted and observed
    values are almost the same and there is no meaning to say if we are
    more sure in actions (by decreasing $Q_t$) or in observations (by
    decreasing $R_t$.
    
    In the second data set the robot again moves in one direction but
    with constant action 30. As the value of action was statically
    specified to be equal to 16, the predicted value does not converge
    to the observed one without modifying the Kalman Filter parameters.
    This happens because we are still equally sure (or equally unsure)
    about both action and observation ($Q_t = R_t = 4.0$).
    
    To change this behavior we can decrease the value of $R_t$ to 1.0 which
    means that we are more sure in observations than actions. This will
    decrease the difference between predicted and observed values.
    However, this difference will remain the same over time. This
    happens because of the statically specified value of action, which
    does not adapt. We still assume at each timestep that the action
    performed by the robot equals 16.
    
  \item Change motion\\
    With these data sets we can observe the change of direction in the
    robot's movement. The situation here is very similar to the perfect
    case in terms of 'trust' in actions or in observations. The
    difference between the observed and predicted values increases
    dramatically when the robot starts to move in the opposite
    direction. The value of statically specified action is still 16 even
    when the robot moves back to the wall. This fact prevents the
    convergence between the observed and predicted values with or
    without the Kalman filter parameters modification.
    
\end{enumerate}

As we have seen during the experiments Kalman filter works very good
when the specified action matches the actual action of the robot.
However, the big limitation of the used model was the static
specification of action. In practice we should adapt it through the
timesteps to get the full power of the Kalman filter.

\section{Detection and Tracking of a moving person}

The problem of detection consists in comparison of obtained laser data
with already stored initial data, and making a conclusion if there is a
moving object and where. 

To solve this problem we store initial laser data in the
\textbf{background[]} array. Then, each frame of laser data is compared
to the background information. If for a particular beam 
\begin{lstlisting}
(background[i] - dataLaserR[i]) > threshold
\end{lstlisting}
then the motion is detected there, and we memorize the number of this
beam in \textbf{detection[]} array. The \textbf{threshold} value is
deterministic and was chosen to be equal to \textbf{15}.

However, this technique is not sufficient to detect a moving object. Due
to the laser's sensibility limitations, it is a common case to detect some
motion when there is actually a wall or some static obstacle. So, to get
rid of these falsely detected beams the following idea was
applied (assuming that we have only one moving object, and this object
is represented by a cluster of at least 4 consecutively detected beams).

Each time we detect a beam with motion we increase a counter. If after a
beam with detection[beam\#]=1 there are more then 30 beams with
detection[]=0 and counter's value is less than 4, then we discard the
counter. This means that the detection was wrong (Figure
\ref{fig:wrong}). If, however, the
counter is bigger than 4 and we detect 30 beams with 0 as a detection
value --- this means that a moving object was found and we can stop
object detection. The reason why we need to count 30 undetected beams is
that in the most cases the sensor does not detect some beams in the
middle of object's position (Figure \ref{fig:undetected}). Examples of such behaviors are shown below:

\begin{figure}[H]
		  \centering
		  \subfloat[Wrongly detected
      motion]{\includegraphics[width=0.4\textwidth]{problems.png}\label{fig:wrong}}
		  \hspace{0.1cm}
		  \subfloat[Undetected
      motion]{\includegraphics[width=0.4\textwidth]{problems2.png}\label{fig:undetected}}
      \hspace{0.1cm}
		  \subfloat[Wrongly detected motion near an object]{\includegraphics[width=0.4\textwidth]{problems3.png}\label{fig:near}}
		  \caption{Problems with object detection}
		  \label{fig:problems}
\end{figure}

Additionally, the problem with wrongly detected motion near an
actually moving object (Figure \ref{fig:near}) was solved.

To determine the position of a moving object the bounding box was built
as can be seen on the figure. The points which define this bounding box
were taken as the coordinates of the first and the last beams of the
detected beams cluster. As an actual \textbf{observation} we take a
center of the bounding box.

As the next step we provide this value of an observation as well as the
value of an \textbf{action} to the Kalman filter. As the movement of an object is
more unpredicted in 2D than in 1D, the statically defined action value
was defined to be small. To emphasize the fact that we are more sure in
observations than in actions, we define $R_t=1$ and $Q_t=4$.

Given this, the Kalman filter produces relatively good results with an
error of 4-10cm.


\end{document}


